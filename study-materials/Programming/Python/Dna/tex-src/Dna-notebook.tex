\documentclass[11pt]{article}

    \usepackage[breakable]{tcolorbox}
    \usepackage{parskip} % Stop auto-indenting (to mimic markdown behaviour)
    
    \usepackage{iftex}
    \ifPDFTeX
    	\usepackage[T1]{fontenc}
    	\usepackage{mathpazo}
    \else
    	\usepackage{fontspec}
    \fi

    % Basic figure setup, for now with no caption control since it's done
    % automatically by Pandoc (which extracts ![](path) syntax from Markdown).
    \usepackage{graphicx}
    % Maintain compatibility with old templates. Remove in nbconvert 6.0
    \let\Oldincludegraphics\includegraphics
    % Ensure that by default, figures have no caption (until we provide a
    % proper Figure object with a Caption API and a way to capture that
    % in the conversion process - todo).
    \usepackage{caption}
    \DeclareCaptionFormat{nocaption}{}
    \captionsetup{format=nocaption,aboveskip=0pt,belowskip=0pt}

    \usepackage{float}
    \floatplacement{figure}{H} % forces figures to be placed at the correct location
    \usepackage{xcolor} % Allow colors to be defined
    \usepackage{enumerate} % Needed for markdown enumerations to work
    \usepackage{geometry} % Used to adjust the document margins
    \usepackage{amsmath} % Equations
    \usepackage{amssymb} % Equations
    \usepackage{textcomp} % defines textquotesingle
    % Hack from http://tex.stackexchange.com/a/47451/13684:
    \AtBeginDocument{%
        \def\PYZsq{\textquotesingle}% Upright quotes in Pygmentized code
    }
    \usepackage{upquote} % Upright quotes for verbatim code
    \usepackage{eurosym} % defines \euro
    \usepackage[mathletters]{ucs} % Extended unicode (utf-8) support
    \usepackage{fancyvrb} % verbatim replacement that allows latex
    \usepackage{grffile} % extends the file name processing of package graphics 
                         % to support a larger range
    \makeatletter % fix for old versions of grffile with XeLaTeX
    \@ifpackagelater{grffile}{2019/11/01}
    {
      % Do nothing on new versions
    }
    {
      \def\Gread@@xetex#1{%
        \IfFileExists{"\Gin@base".bb}%
        {\Gread@eps{\Gin@base.bb}}%
        {\Gread@@xetex@aux#1}%
      }
    }
    \makeatother
    \usepackage[Export]{adjustbox} % Used to constrain images to a maximum size
    \adjustboxset{max size={0.9\linewidth}{0.9\paperheight}}

    % The hyperref package gives us a pdf with properly built
    % internal navigation ('pdf bookmarks' for the table of contents,
    % internal cross-reference links, web links for URLs, etc.)
    \usepackage{hyperref}
    % The default LaTeX title has an obnoxious amount of whitespace. By default,
    % titling removes some of it. It also provides customization options.
    \usepackage{titling}
    \usepackage{longtable} % longtable support required by pandoc >1.10
    \usepackage{booktabs}  % table support for pandoc > 1.12.2
    \usepackage[inline]{enumitem} % IRkernel/repr support (it uses the enumerate* environment)
    \usepackage[normalem]{ulem} % ulem is needed to support strikethroughs (\sout)
                                % normalem makes italics be italics, not underlines
    \usepackage{mathrsfs}
    

    
    % Colors for the hyperref package
    \definecolor{urlcolor}{rgb}{0,.145,.698}
    \definecolor{linkcolor}{rgb}{.71,0.21,0.01}
    \definecolor{citecolor}{rgb}{.12,.54,.11}

    % ANSI colors
    \definecolor{ansi-black}{HTML}{3E424D}
    \definecolor{ansi-black-intense}{HTML}{282C36}
    \definecolor{ansi-red}{HTML}{E75C58}
    \definecolor{ansi-red-intense}{HTML}{B22B31}
    \definecolor{ansi-green}{HTML}{00A250}
    \definecolor{ansi-green-intense}{HTML}{007427}
    \definecolor{ansi-yellow}{HTML}{DDB62B}
    \definecolor{ansi-yellow-intense}{HTML}{B27D12}
    \definecolor{ansi-blue}{HTML}{208FFB}
    \definecolor{ansi-blue-intense}{HTML}{0065CA}
    \definecolor{ansi-magenta}{HTML}{D160C4}
    \definecolor{ansi-magenta-intense}{HTML}{A03196}
    \definecolor{ansi-cyan}{HTML}{60C6C8}
    \definecolor{ansi-cyan-intense}{HTML}{258F8F}
    \definecolor{ansi-white}{HTML}{C5C1B4}
    \definecolor{ansi-white-intense}{HTML}{A1A6B2}
    \definecolor{ansi-default-inverse-fg}{HTML}{FFFFFF}
    \definecolor{ansi-default-inverse-bg}{HTML}{000000}

    % common color for the border for error outputs.
    \definecolor{outerrorbackground}{HTML}{FFDFDF}

    % commands and environments needed by pandoc snippets
    % extracted from the output of `pandoc -s`
    \providecommand{\tightlist}{%
      \setlength{\itemsep}{0pt}\setlength{\parskip}{0pt}}
    \DefineVerbatimEnvironment{Highlighting}{Verbatim}{commandchars=\\\{\}}
    % Add ',fontsize=\small' for more characters per line
    \newenvironment{Shaded}{}{}
    \newcommand{\KeywordTok}[1]{\textcolor[rgb]{0.00,0.44,0.13}{\textbf{{#1}}}}
    \newcommand{\DataTypeTok}[1]{\textcolor[rgb]{0.56,0.13,0.00}{{#1}}}
    \newcommand{\DecValTok}[1]{\textcolor[rgb]{0.25,0.63,0.44}{{#1}}}
    \newcommand{\BaseNTok}[1]{\textcolor[rgb]{0.25,0.63,0.44}{{#1}}}
    \newcommand{\FloatTok}[1]{\textcolor[rgb]{0.25,0.63,0.44}{{#1}}}
    \newcommand{\CharTok}[1]{\textcolor[rgb]{0.25,0.44,0.63}{{#1}}}
    \newcommand{\StringTok}[1]{\textcolor[rgb]{0.25,0.44,0.63}{{#1}}}
    \newcommand{\CommentTok}[1]{\textcolor[rgb]{0.38,0.63,0.69}{\textit{{#1}}}}
    \newcommand{\OtherTok}[1]{\textcolor[rgb]{0.00,0.44,0.13}{{#1}}}
    \newcommand{\AlertTok}[1]{\textcolor[rgb]{1.00,0.00,0.00}{\textbf{{#1}}}}
    \newcommand{\FunctionTok}[1]{\textcolor[rgb]{0.02,0.16,0.49}{{#1}}}
    \newcommand{\RegionMarkerTok}[1]{{#1}}
    \newcommand{\ErrorTok}[1]{\textcolor[rgb]{1.00,0.00,0.00}{\textbf{{#1}}}}
    \newcommand{\NormalTok}[1]{{#1}}
    
    % Additional commands for more recent versions of Pandoc
    \newcommand{\ConstantTok}[1]{\textcolor[rgb]{0.53,0.00,0.00}{{#1}}}
    \newcommand{\SpecialCharTok}[1]{\textcolor[rgb]{0.25,0.44,0.63}{{#1}}}
    \newcommand{\VerbatimStringTok}[1]{\textcolor[rgb]{0.25,0.44,0.63}{{#1}}}
    \newcommand{\SpecialStringTok}[1]{\textcolor[rgb]{0.73,0.40,0.53}{{#1}}}
    \newcommand{\ImportTok}[1]{{#1}}
    \newcommand{\DocumentationTok}[1]{\textcolor[rgb]{0.73,0.13,0.13}{\textit{{#1}}}}
    \newcommand{\AnnotationTok}[1]{\textcolor[rgb]{0.38,0.63,0.69}{\textbf{\textit{{#1}}}}}
    \newcommand{\CommentVarTok}[1]{\textcolor[rgb]{0.38,0.63,0.69}{\textbf{\textit{{#1}}}}}
    \newcommand{\VariableTok}[1]{\textcolor[rgb]{0.10,0.09,0.49}{{#1}}}
    \newcommand{\ControlFlowTok}[1]{\textcolor[rgb]{0.00,0.44,0.13}{\textbf{{#1}}}}
    \newcommand{\OperatorTok}[1]{\textcolor[rgb]{0.40,0.40,0.40}{{#1}}}
    \newcommand{\BuiltInTok}[1]{{#1}}
    \newcommand{\ExtensionTok}[1]{{#1}}
    \newcommand{\PreprocessorTok}[1]{\textcolor[rgb]{0.74,0.48,0.00}{{#1}}}
    \newcommand{\AttributeTok}[1]{\textcolor[rgb]{0.49,0.56,0.16}{{#1}}}
    \newcommand{\InformationTok}[1]{\textcolor[rgb]{0.38,0.63,0.69}{\textbf{\textit{{#1}}}}}
    \newcommand{\WarningTok}[1]{\textcolor[rgb]{0.38,0.63,0.69}{\textbf{\textit{{#1}}}}}
    
    
    % Define a nice break command that doesn't care if a line doesn't already
    % exist.
    \def\br{\hspace*{\fill} \\* }
    % Math Jax compatibility definitions
    \def\gt{>}
    \def\lt{<}
    \let\Oldtex\TeX
    \let\Oldlatex\LaTeX
    \renewcommand{\TeX}{\textrm{\Oldtex}}
    \renewcommand{\LaTeX}{\textrm{\Oldlatex}}
    % Document parameters
    % Document title
    \title{Dna-notebook}
    
    
    
    
    
% Pygments definitions
\makeatletter
\def\PY@reset{\let\PY@it=\relax \let\PY@bf=\relax%
    \let\PY@ul=\relax \let\PY@tc=\relax%
    \let\PY@bc=\relax \let\PY@ff=\relax}
\def\PY@tok#1{\csname PY@tok@#1\endcsname}
\def\PY@toks#1+{\ifx\relax#1\empty\else%
    \PY@tok{#1}\expandafter\PY@toks\fi}
\def\PY@do#1{\PY@bc{\PY@tc{\PY@ul{%
    \PY@it{\PY@bf{\PY@ff{#1}}}}}}}
\def\PY#1#2{\PY@reset\PY@toks#1+\relax+\PY@do{#2}}

\@namedef{PY@tok@w}{\def\PY@tc##1{\textcolor[rgb]{0.73,0.73,0.73}{##1}}}
\@namedef{PY@tok@c}{\let\PY@it=\textit\def\PY@tc##1{\textcolor[rgb]{0.25,0.50,0.50}{##1}}}
\@namedef{PY@tok@cp}{\def\PY@tc##1{\textcolor[rgb]{0.74,0.48,0.00}{##1}}}
\@namedef{PY@tok@k}{\let\PY@bf=\textbf\def\PY@tc##1{\textcolor[rgb]{0.00,0.50,0.00}{##1}}}
\@namedef{PY@tok@kp}{\def\PY@tc##1{\textcolor[rgb]{0.00,0.50,0.00}{##1}}}
\@namedef{PY@tok@kt}{\def\PY@tc##1{\textcolor[rgb]{0.69,0.00,0.25}{##1}}}
\@namedef{PY@tok@o}{\def\PY@tc##1{\textcolor[rgb]{0.40,0.40,0.40}{##1}}}
\@namedef{PY@tok@ow}{\let\PY@bf=\textbf\def\PY@tc##1{\textcolor[rgb]{0.67,0.13,1.00}{##1}}}
\@namedef{PY@tok@nb}{\def\PY@tc##1{\textcolor[rgb]{0.00,0.50,0.00}{##1}}}
\@namedef{PY@tok@nf}{\def\PY@tc##1{\textcolor[rgb]{0.00,0.00,1.00}{##1}}}
\@namedef{PY@tok@nc}{\let\PY@bf=\textbf\def\PY@tc##1{\textcolor[rgb]{0.00,0.00,1.00}{##1}}}
\@namedef{PY@tok@nn}{\let\PY@bf=\textbf\def\PY@tc##1{\textcolor[rgb]{0.00,0.00,1.00}{##1}}}
\@namedef{PY@tok@ne}{\let\PY@bf=\textbf\def\PY@tc##1{\textcolor[rgb]{0.82,0.25,0.23}{##1}}}
\@namedef{PY@tok@nv}{\def\PY@tc##1{\textcolor[rgb]{0.10,0.09,0.49}{##1}}}
\@namedef{PY@tok@no}{\def\PY@tc##1{\textcolor[rgb]{0.53,0.00,0.00}{##1}}}
\@namedef{PY@tok@nl}{\def\PY@tc##1{\textcolor[rgb]{0.63,0.63,0.00}{##1}}}
\@namedef{PY@tok@ni}{\let\PY@bf=\textbf\def\PY@tc##1{\textcolor[rgb]{0.60,0.60,0.60}{##1}}}
\@namedef{PY@tok@na}{\def\PY@tc##1{\textcolor[rgb]{0.49,0.56,0.16}{##1}}}
\@namedef{PY@tok@nt}{\let\PY@bf=\textbf\def\PY@tc##1{\textcolor[rgb]{0.00,0.50,0.00}{##1}}}
\@namedef{PY@tok@nd}{\def\PY@tc##1{\textcolor[rgb]{0.67,0.13,1.00}{##1}}}
\@namedef{PY@tok@s}{\def\PY@tc##1{\textcolor[rgb]{0.73,0.13,0.13}{##1}}}
\@namedef{PY@tok@sd}{\let\PY@it=\textit\def\PY@tc##1{\textcolor[rgb]{0.73,0.13,0.13}{##1}}}
\@namedef{PY@tok@si}{\let\PY@bf=\textbf\def\PY@tc##1{\textcolor[rgb]{0.73,0.40,0.53}{##1}}}
\@namedef{PY@tok@se}{\let\PY@bf=\textbf\def\PY@tc##1{\textcolor[rgb]{0.73,0.40,0.13}{##1}}}
\@namedef{PY@tok@sr}{\def\PY@tc##1{\textcolor[rgb]{0.73,0.40,0.53}{##1}}}
\@namedef{PY@tok@ss}{\def\PY@tc##1{\textcolor[rgb]{0.10,0.09,0.49}{##1}}}
\@namedef{PY@tok@sx}{\def\PY@tc##1{\textcolor[rgb]{0.00,0.50,0.00}{##1}}}
\@namedef{PY@tok@m}{\def\PY@tc##1{\textcolor[rgb]{0.40,0.40,0.40}{##1}}}
\@namedef{PY@tok@gh}{\let\PY@bf=\textbf\def\PY@tc##1{\textcolor[rgb]{0.00,0.00,0.50}{##1}}}
\@namedef{PY@tok@gu}{\let\PY@bf=\textbf\def\PY@tc##1{\textcolor[rgb]{0.50,0.00,0.50}{##1}}}
\@namedef{PY@tok@gd}{\def\PY@tc##1{\textcolor[rgb]{0.63,0.00,0.00}{##1}}}
\@namedef{PY@tok@gi}{\def\PY@tc##1{\textcolor[rgb]{0.00,0.63,0.00}{##1}}}
\@namedef{PY@tok@gr}{\def\PY@tc##1{\textcolor[rgb]{1.00,0.00,0.00}{##1}}}
\@namedef{PY@tok@ge}{\let\PY@it=\textit}
\@namedef{PY@tok@gs}{\let\PY@bf=\textbf}
\@namedef{PY@tok@gp}{\let\PY@bf=\textbf\def\PY@tc##1{\textcolor[rgb]{0.00,0.00,0.50}{##1}}}
\@namedef{PY@tok@go}{\def\PY@tc##1{\textcolor[rgb]{0.53,0.53,0.53}{##1}}}
\@namedef{PY@tok@gt}{\def\PY@tc##1{\textcolor[rgb]{0.00,0.27,0.87}{##1}}}
\@namedef{PY@tok@err}{\def\PY@bc##1{{\setlength{\fboxsep}{\string -\fboxrule}\fcolorbox[rgb]{1.00,0.00,0.00}{1,1,1}{\strut ##1}}}}
\@namedef{PY@tok@kc}{\let\PY@bf=\textbf\def\PY@tc##1{\textcolor[rgb]{0.00,0.50,0.00}{##1}}}
\@namedef{PY@tok@kd}{\let\PY@bf=\textbf\def\PY@tc##1{\textcolor[rgb]{0.00,0.50,0.00}{##1}}}
\@namedef{PY@tok@kn}{\let\PY@bf=\textbf\def\PY@tc##1{\textcolor[rgb]{0.00,0.50,0.00}{##1}}}
\@namedef{PY@tok@kr}{\let\PY@bf=\textbf\def\PY@tc##1{\textcolor[rgb]{0.00,0.50,0.00}{##1}}}
\@namedef{PY@tok@bp}{\def\PY@tc##1{\textcolor[rgb]{0.00,0.50,0.00}{##1}}}
\@namedef{PY@tok@fm}{\def\PY@tc##1{\textcolor[rgb]{0.00,0.00,1.00}{##1}}}
\@namedef{PY@tok@vc}{\def\PY@tc##1{\textcolor[rgb]{0.10,0.09,0.49}{##1}}}
\@namedef{PY@tok@vg}{\def\PY@tc##1{\textcolor[rgb]{0.10,0.09,0.49}{##1}}}
\@namedef{PY@tok@vi}{\def\PY@tc##1{\textcolor[rgb]{0.10,0.09,0.49}{##1}}}
\@namedef{PY@tok@vm}{\def\PY@tc##1{\textcolor[rgb]{0.10,0.09,0.49}{##1}}}
\@namedef{PY@tok@sa}{\def\PY@tc##1{\textcolor[rgb]{0.73,0.13,0.13}{##1}}}
\@namedef{PY@tok@sb}{\def\PY@tc##1{\textcolor[rgb]{0.73,0.13,0.13}{##1}}}
\@namedef{PY@tok@sc}{\def\PY@tc##1{\textcolor[rgb]{0.73,0.13,0.13}{##1}}}
\@namedef{PY@tok@dl}{\def\PY@tc##1{\textcolor[rgb]{0.73,0.13,0.13}{##1}}}
\@namedef{PY@tok@s2}{\def\PY@tc##1{\textcolor[rgb]{0.73,0.13,0.13}{##1}}}
\@namedef{PY@tok@sh}{\def\PY@tc##1{\textcolor[rgb]{0.73,0.13,0.13}{##1}}}
\@namedef{PY@tok@s1}{\def\PY@tc##1{\textcolor[rgb]{0.73,0.13,0.13}{##1}}}
\@namedef{PY@tok@mb}{\def\PY@tc##1{\textcolor[rgb]{0.40,0.40,0.40}{##1}}}
\@namedef{PY@tok@mf}{\def\PY@tc##1{\textcolor[rgb]{0.40,0.40,0.40}{##1}}}
\@namedef{PY@tok@mh}{\def\PY@tc##1{\textcolor[rgb]{0.40,0.40,0.40}{##1}}}
\@namedef{PY@tok@mi}{\def\PY@tc##1{\textcolor[rgb]{0.40,0.40,0.40}{##1}}}
\@namedef{PY@tok@il}{\def\PY@tc##1{\textcolor[rgb]{0.40,0.40,0.40}{##1}}}
\@namedef{PY@tok@mo}{\def\PY@tc##1{\textcolor[rgb]{0.40,0.40,0.40}{##1}}}
\@namedef{PY@tok@ch}{\let\PY@it=\textit\def\PY@tc##1{\textcolor[rgb]{0.25,0.50,0.50}{##1}}}
\@namedef{PY@tok@cm}{\let\PY@it=\textit\def\PY@tc##1{\textcolor[rgb]{0.25,0.50,0.50}{##1}}}
\@namedef{PY@tok@cpf}{\let\PY@it=\textit\def\PY@tc##1{\textcolor[rgb]{0.25,0.50,0.50}{##1}}}
\@namedef{PY@tok@c1}{\let\PY@it=\textit\def\PY@tc##1{\textcolor[rgb]{0.25,0.50,0.50}{##1}}}
\@namedef{PY@tok@cs}{\let\PY@it=\textit\def\PY@tc##1{\textcolor[rgb]{0.25,0.50,0.50}{##1}}}

\def\PYZbs{\char`\\}
\def\PYZus{\char`\_}
\def\PYZob{\char`\{}
\def\PYZcb{\char`\}}
\def\PYZca{\char`\^}
\def\PYZam{\char`\&}
\def\PYZlt{\char`\<}
\def\PYZgt{\char`\>}
\def\PYZsh{\char`\#}
\def\PYZpc{\char`\%}
\def\PYZdl{\char`\$}
\def\PYZhy{\char`\-}
\def\PYZsq{\char`\'}
\def\PYZdq{\char`\"}
\def\PYZti{\char`\~}
% for compatibility with earlier versions
\def\PYZat{@}
\def\PYZlb{[}
\def\PYZrb{]}
\makeatother


    % For linebreaks inside Verbatim environment from package fancyvrb. 
    \makeatletter
        \newbox\Wrappedcontinuationbox 
        \newbox\Wrappedvisiblespacebox 
        \newcommand*\Wrappedvisiblespace {\textcolor{red}{\textvisiblespace}} 
        \newcommand*\Wrappedcontinuationsymbol {\textcolor{red}{\llap{\tiny$\m@th\hookrightarrow$}}} 
        \newcommand*\Wrappedcontinuationindent {3ex } 
        \newcommand*\Wrappedafterbreak {\kern\Wrappedcontinuationindent\copy\Wrappedcontinuationbox} 
        % Take advantage of the already applied Pygments mark-up to insert 
        % potential linebreaks for TeX processing. 
        %        {, <, #, %, $, ' and ": go to next line. 
        %        _, }, ^, &, >, - and ~: stay at end of broken line. 
        % Use of \textquotesingle for straight quote. 
        \newcommand*\Wrappedbreaksatspecials {% 
            \def\PYGZus{\discretionary{\char`\_}{\Wrappedafterbreak}{\char`\_}}% 
            \def\PYGZob{\discretionary{}{\Wrappedafterbreak\char`\{}{\char`\{}}% 
            \def\PYGZcb{\discretionary{\char`\}}{\Wrappedafterbreak}{\char`\}}}% 
            \def\PYGZca{\discretionary{\char`\^}{\Wrappedafterbreak}{\char`\^}}% 
            \def\PYGZam{\discretionary{\char`\&}{\Wrappedafterbreak}{\char`\&}}% 
            \def\PYGZlt{\discretionary{}{\Wrappedafterbreak\char`\<}{\char`\<}}% 
            \def\PYGZgt{\discretionary{\char`\>}{\Wrappedafterbreak}{\char`\>}}% 
            \def\PYGZsh{\discretionary{}{\Wrappedafterbreak\char`\#}{\char`\#}}% 
            \def\PYGZpc{\discretionary{}{\Wrappedafterbreak\char`\%}{\char`\%}}% 
            \def\PYGZdl{\discretionary{}{\Wrappedafterbreak\char`\$}{\char`\$}}% 
            \def\PYGZhy{\discretionary{\char`\-}{\Wrappedafterbreak}{\char`\-}}% 
            \def\PYGZsq{\discretionary{}{\Wrappedafterbreak\textquotesingle}{\textquotesingle}}% 
            \def\PYGZdq{\discretionary{}{\Wrappedafterbreak\char`\"}{\char`\"}}% 
            \def\PYGZti{\discretionary{\char`\~}{\Wrappedafterbreak}{\char`\~}}% 
        } 
        % Some characters . , ; ? ! / are not pygmentized. 
        % This macro makes them "active" and they will insert potential linebreaks 
        \newcommand*\Wrappedbreaksatpunct {% 
            \lccode`\~`\.\lowercase{\def~}{\discretionary{\hbox{\char`\.}}{\Wrappedafterbreak}{\hbox{\char`\.}}}% 
            \lccode`\~`\,\lowercase{\def~}{\discretionary{\hbox{\char`\,}}{\Wrappedafterbreak}{\hbox{\char`\,}}}% 
            \lccode`\~`\;\lowercase{\def~}{\discretionary{\hbox{\char`\;}}{\Wrappedafterbreak}{\hbox{\char`\;}}}% 
            \lccode`\~`\:\lowercase{\def~}{\discretionary{\hbox{\char`\:}}{\Wrappedafterbreak}{\hbox{\char`\:}}}% 
            \lccode`\~`\?\lowercase{\def~}{\discretionary{\hbox{\char`\?}}{\Wrappedafterbreak}{\hbox{\char`\?}}}% 
            \lccode`\~`\!\lowercase{\def~}{\discretionary{\hbox{\char`\!}}{\Wrappedafterbreak}{\hbox{\char`\!}}}% 
            \lccode`\~`\/\lowercase{\def~}{\discretionary{\hbox{\char`\/}}{\Wrappedafterbreak}{\hbox{\char`\/}}}% 
            \catcode`\.\active
            \catcode`\,\active 
            \catcode`\;\active
            \catcode`\:\active
            \catcode`\?\active
            \catcode`\!\active
            \catcode`\/\active 
            \lccode`\~`\~ 	
        }
    \makeatother

    \let\OriginalVerbatim=\Verbatim
    \makeatletter
    \renewcommand{\Verbatim}[1][1]{%
        %\parskip\z@skip
        \sbox\Wrappedcontinuationbox {\Wrappedcontinuationsymbol}%
        \sbox\Wrappedvisiblespacebox {\FV@SetupFont\Wrappedvisiblespace}%
        \def\FancyVerbFormatLine ##1{\hsize\linewidth
            \vtop{\raggedright\hyphenpenalty\z@\exhyphenpenalty\z@
                \doublehyphendemerits\z@\finalhyphendemerits\z@
                \strut ##1\strut}%
        }%
        % If the linebreak is at a space, the latter will be displayed as visible
        % space at end of first line, and a continuation symbol starts next line.
        % Stretch/shrink are however usually zero for typewriter font.
        \def\FV@Space {%
            \nobreak\hskip\z@ plus\fontdimen3\font minus\fontdimen4\font
            \discretionary{\copy\Wrappedvisiblespacebox}{\Wrappedafterbreak}
            {\kern\fontdimen2\font}%
        }%
        
        % Allow breaks at special characters using \PYG... macros.
        \Wrappedbreaksatspecials
        % Breaks at punctuation characters . , ; ? ! and / need catcode=\active 	
        \OriginalVerbatim[#1,codes*=\Wrappedbreaksatpunct]%
    }
    \makeatother

    % Exact colors from NB
    \definecolor{incolor}{HTML}{303F9F}
    \definecolor{outcolor}{HTML}{D84315}
    \definecolor{cellborder}{HTML}{CFCFCF}
    \definecolor{cellbackground}{HTML}{F7F7F7}
    
    % prompt
    \makeatletter
    \newcommand{\boxspacing}{\kern\kvtcb@left@rule\kern\kvtcb@boxsep}
    \makeatother
    \newcommand{\prompt}[4]{
        {\ttfamily\llap{{\color{#2}[#3]:\hspace{3pt}#4}}\vspace{-\baselineskip}}
    }
    

    
    % Prevent overflowing lines due to hard-to-break entities
    \sloppy 
    % Setup hyperref package
    \hypersetup{
      breaklinks=true,  % so long urls are correctly broken across lines
      colorlinks=true,
      urlcolor=urlcolor,
      linkcolor=linkcolor,
      citecolor=citecolor,
      }
    % Slightly bigger margins than the latex defaults
    
    \geometry{verbose,tmargin=1in,bmargin=1in,lmargin=1in,rmargin=1in}
    
    

\begin{document}
    
    \maketitle
    
    

    
    \hypertarget{dna-identification-in-python}{%
\section{DNA Identification in
Python}\label{dna-identification-in-python}}

\hypertarget{a-program-capable-of-identifying-a-person-based-on-their-dna.}{%
\subsection{A program capable of identifying a person based on their
DNA.}\label{a-program-capable-of-identifying-a-person-based-on-their-dna.}}

    \begin{tcolorbox}[breakable, size=fbox, boxrule=1pt, pad at break*=1mm,colback=cellbackground, colframe=cellborder]
\prompt{In}{incolor}{1}{\boxspacing}
\begin{Verbatim}[commandchars=\\\{\}]
\PY{c+c1}{\PYZsh{} Import libraries to be used}
\PY{k+kn}{import} \PY{n+nn}{csv}

\PY{k+kn}{from} \PY{n+nn}{sys} \PY{k+kn}{import} \PY{n}{argv}\PY{p}{,} \PY{n}{exit}
\end{Verbatim}
\end{tcolorbox}

    \textbf{NOTE:} Using \emph{manually generated} \texttt{argv} of type
list of arguments instead of \texttt{sys.argv} for simplicity.

\href{https://gist.github.com/gbishop/acf40b86a9bca2d571fa}{Jupyter
Notebook and Command-line arguments possible solution}

    \begin{tcolorbox}[breakable, size=fbox, boxrule=1pt, pad at break*=1mm,colback=cellbackground, colframe=cellborder]
\prompt{In}{incolor}{2}{\boxspacing}
\begin{Verbatim}[commandchars=\\\{\}]
\PY{n}{argv}\PY{p}{:} \PY{n+nb}{list} \PY{o}{=} \PY{p}{[}\PY{l+s+s2}{\PYZdq{}}\PY{l+s+s2}{dna.py}\PY{l+s+s2}{\PYZdq{}}\PY{p}{,} \PY{n+nb}{input}\PY{p}{(}\PY{l+s+s2}{\PYZdq{}}\PY{l+s+s2}{Database: }\PY{l+s+s2}{\PYZdq{}}\PY{p}{)}\PY{p}{,} \PY{n+nb}{input}\PY{p}{(}\PY{l+s+s2}{\PYZdq{}}\PY{l+s+s2}{Sequence: }\PY{l+s+s2}{\PYZdq{}}\PY{p}{)}\PY{p}{]}
\end{Verbatim}
\end{tcolorbox}

    \hypertarget{using-command-line-arguments}{%
\subsubsection{Using command line
arguments}\label{using-command-line-arguments}}

\textbf{Usage:}

\begin{Shaded}
\begin{Highlighting}[]
\NormalTok{$ python3 dna.py $DATABASE $SEQUENCE}
\end{Highlighting}
\end{Shaded}

    \begin{tcolorbox}[breakable, size=fbox, boxrule=1pt, pad at break*=1mm,colback=cellbackground, colframe=cellborder]
\prompt{In}{incolor}{3}{\boxspacing}
\begin{Verbatim}[commandchars=\\\{\}]
\PY{k}{if} \PY{n+nb}{len}\PY{p}{(}\PY{n}{argv}\PY{p}{)} \PY{o}{!=} \PY{l+m+mi}{3}\PY{p}{:}
    \PY{n+nb}{print}\PY{p}{(}\PY{l+s+s2}{\PYZdq{}}\PY{l+s+s2}{Usage: python3 dna.py \PYZdl{}DATABASE \PYZdl{}SEQUENCE}\PY{l+s+s2}{\PYZdq{}}\PY{p}{)}

    \PY{c+c1}{\PYZsh{} Exit with an error and code 1}
    \PY{n}{exit}\PY{p}{(}\PY{l+m+mi}{1}\PY{p}{)}
\end{Verbatim}
\end{tcolorbox}

    \hypertarget{a-check-is-needed-to-determine-valid-file-selection}{%
\subsubsection{A check is needed to determine valid file
selection}\label{a-check-is-needed-to-determine-valid-file-selection}}

\begin{enumerate}
\def\labelenumi{\arabic{enumi}.}
\item
  \textbf{Command line-argument} - \texttt{database\_path} of type
  \texttt{.csv}
\item
  \textbf{Command line-argument} - \texttt{sequence\_path} of type
  \texttt{.txt}
\end{enumerate}

\emph{Otherwise error}

    \begin{tcolorbox}[breakable, size=fbox, boxrule=1pt, pad at break*=1mm,colback=cellbackground, colframe=cellborder]
\prompt{In}{incolor}{4}{\boxspacing}
\begin{Verbatim}[commandchars=\\\{\}]
\PY{n}{database\PYZus{}path}\PY{p}{,} \PY{n}{sequence\PYZus{}path} \PY{o}{=} \PY{n}{argv}\PY{p}{[}\PY{l+m+mi}{1}\PY{p}{]}\PY{p}{,} \PY{n}{argv}\PY{p}{[}\PY{l+m+mi}{2}\PY{p}{]}
\PY{n}{file\PYZus{}ext1}\PY{p}{,} \PY{n}{file\PYZus{}ext2} \PY{o}{=} \PY{n}{database\PYZus{}path}\PY{o}{.}\PY{n}{split}\PY{p}{(}\PY{l+s+s2}{\PYZdq{}}\PY{l+s+s2}{.}\PY{l+s+s2}{\PYZdq{}}\PY{p}{)}\PY{p}{[}\PY{l+m+mi}{1}\PY{p}{]}\PY{p}{,} \PY{n}{sequence\PYZus{}path}\PY{o}{.}\PY{n}{split}\PY{p}{(}\PY{l+s+s2}{\PYZdq{}}\PY{l+s+s2}{.}\PY{l+s+s2}{\PYZdq{}}\PY{p}{)}\PY{p}{[}\PY{l+m+mi}{1}\PY{p}{]}
\end{Verbatim}
\end{tcolorbox}

    \hypertarget{a-check-for-valid-file-extensions}{%
\subsubsection{A check for valid file
extensions}\label{a-check-for-valid-file-extensions}}

    \begin{tcolorbox}[breakable, size=fbox, boxrule=1pt, pad at break*=1mm,colback=cellbackground, colframe=cellborder]
\prompt{In}{incolor}{5}{\boxspacing}
\begin{Verbatim}[commandchars=\\\{\}]
\PY{k}{if} \PY{n}{file\PYZus{}ext1} \PY{o}{!=} \PY{l+s+s2}{\PYZdq{}}\PY{l+s+s2}{csv}\PY{l+s+s2}{\PYZdq{}} \PY{o+ow}{or} \PY{n}{file\PYZus{}ext2} \PY{o}{!=} \PY{l+s+s2}{\PYZdq{}}\PY{l+s+s2}{txt}\PY{l+s+s2}{\PYZdq{}}\PY{p}{:}
    \PY{n+nb}{print}\PY{p}{(}\PY{l+s+s2}{\PYZdq{}}\PY{l+s+s2}{Usage: python3 dna.py \PYZdl{}DATABASE[.csv] \PYZdl{}SEQUENCE[.txt]}\PY{l+s+s2}{\PYZdq{}}\PY{p}{)}

    \PY{c+c1}{\PYZsh{} Exit with an error and code 1}
    \PY{n}{exit}\PY{p}{(}\PY{l+m+mi}{1}\PY{p}{)}
\end{Verbatim}
\end{tcolorbox}

    \hypertarget{define-a-function-to-prompt-the-user-with-an-error-message-of-files-path}{%
\subsubsection{Define a function to prompt the user with an error
message of file's
path}\label{define-a-function-to-prompt-the-user-with-an-error-message-of-files-path}}

    \begin{tcolorbox}[breakable, size=fbox, boxrule=1pt, pad at break*=1mm,colback=cellbackground, colframe=cellborder]
\prompt{In}{incolor}{6}{\boxspacing}
\begin{Verbatim}[commandchars=\\\{\}]
\PY{k}{def} \PY{n+nf}{file\PYZus{}error}\PY{p}{(}\PY{n}{path}\PY{p}{)}\PY{p}{:}
    \PY{n+nb}{print}\PY{p}{(}\PY{l+s+sa}{f}\PY{l+s+s2}{\PYZdq{}}\PY{l+s+s2}{Could not open }\PY{l+s+si}{\PYZob{}}\PY{n}{path}\PY{l+s+si}{\PYZcb{}}\PY{l+s+s2}{\PYZdq{}}\PY{p}{)}
    \PY{n}{exit}\PY{p}{(}\PY{l+m+mi}{1}\PY{p}{)}
\end{Verbatim}
\end{tcolorbox}

    \hypertarget{compute-function-is-supposd-to-return-the-longest-chain-created-from-a-given-nucleotide}{%
\subsubsection{Compute function is supposd to return the longest chain
created from a given
nucleotide}\label{compute-function-is-supposd-to-return-the-longest-chain-created-from-a-given-nucleotide}}

    \begin{tcolorbox}[breakable, size=fbox, boxrule=1pt, pad at break*=1mm,colback=cellbackground, colframe=cellborder]
\prompt{In}{incolor}{7}{\boxspacing}
\begin{Verbatim}[commandchars=\\\{\}]
\PY{c+c1}{\PYZsh{} Create a function to perform the computations of alike chains of STRs}
\PY{k}{def} \PY{n+nf}{compute}\PY{p}{(}\PY{n}{sequence}\PY{p}{,} \PY{n}{nucleotide}\PY{p}{)} \PY{o}{\PYZhy{}}\PY{o}{\PYZgt{}} \PY{n+nb}{int}\PY{p}{:}

    \PY{c+c1}{\PYZsh{} Get the length of the corresponding nucleotide}
    \PY{n}{n} \PY{o}{=} \PY{n+nb}{len}\PY{p}{(}\PY{n}{nucleotide}\PY{p}{)}

    \PY{c+c1}{\PYZsh{} Create a list to store all computed sums of a specific nucleotide (default value of 0 since not counter could}
    \PY{c+c1}{\PYZsh{} be computed \PYZhy{} cause of crash)}
    \PY{n}{counters}\PY{p}{:} \PY{n+nb}{list} \PY{o}{=} \PY{p}{[}\PY{l+m+mi}{0}\PY{p}{]}

    \PY{c+c1}{\PYZsh{} Iterate through all chars in the sequence \PYZhy{} n}
    \PY{k}{for} \PY{n}{i} \PY{o+ow}{in} \PY{n+nb}{range}\PY{p}{(}\PY{n+nb}{len}\PY{p}{(}\PY{n}{sequence}\PY{p}{)} \PY{o}{\PYZhy{}} \PY{n}{n}\PY{p}{)}\PY{p}{:}

        \PY{c+c1}{\PYZsh{} Set index to 0}
        \PY{n}{index} \PY{o}{=} \PY{l+m+mi}{0}

        \PY{c+c1}{\PYZsh{} Check for head of a sequence \PYZhy{}\PYZgt{} start of a STR chain, a substring method}
        \PY{k}{if} \PY{n}{sequence}\PY{p}{[}\PY{n}{i}\PY{p}{:}\PY{n}{i} \PY{o}{+} \PY{n}{n}\PY{p}{]} \PY{o}{==} \PY{n}{nucleotide}\PY{p}{:}

            \PY{c+c1}{\PYZsh{} Create a counter and set it to 1 (count in the header too)}
            \PY{n}{counter}\PY{p}{:} \PY{n+nb}{int} \PY{o}{=} \PY{l+m+mi}{1}

            \PY{c+c1}{\PYZsh{} Create a while loop to search for continuous chain following the header}
            \PY{k}{while} \PY{k+kc}{True}\PY{p}{:}

                \PY{c+c1}{\PYZsh{} Increment the index by the length of the nucleotide}
                \PY{n}{index} \PY{o}{+}\PY{o}{=} \PY{n}{n}

                \PY{c+c1}{\PYZsh{} Check whether the chain is continuous and not to pass the max. index (a corner case to prevent crash)}
                \PY{k}{if} \PY{n}{sequence}\PY{p}{[}\PY{n}{i} \PY{o}{+} \PY{n}{index}\PY{p}{:} \PY{n}{i} \PY{o}{+} \PY{n}{index} \PY{o}{+} \PY{n}{n}\PY{p}{]} \PY{o}{==} \PY{n}{nucleotide} \PY{o+ow}{and} \PY{p}{(}\PY{n}{i} \PY{o}{+} \PY{n}{index}\PY{p}{)} \PY{o}{\PYZlt{}} \PY{n+nb}{len}\PY{p}{(}\PY{n}{sequence}\PY{p}{)}\PY{p}{:}

                    \PY{c+c1}{\PYZsh{} Increment the counter for this chain}
                    \PY{n}{counter} \PY{o}{+}\PY{o}{=} \PY{l+m+mi}{1}

                \PY{c+c1}{\PYZsh{} Otherwise break out of the loop and find another head}
                \PY{k}{else}\PY{p}{:}
                    \PY{k}{break}

            \PY{c+c1}{\PYZsh{} Store all found chains (their lengths) corresponding to a particular nucleotide in a list}
            \PY{n}{counters}\PY{o}{.}\PY{n}{append}\PY{p}{(}\PY{n}{counter}\PY{p}{)}

    \PY{c+c1}{\PYZsh{} Return the biggest of them}
    \PY{k}{return} \PY{n+nb}{max}\PY{p}{(}\PY{n}{counters}\PY{p}{)}
\end{Verbatim}
\end{tcolorbox}

    \hypertarget{check-function-is-supposed-to-return-the-name-of-a-person-or-none-if-not-found}{%
\subsubsection{Check function is supposed to return the name of a person
or None (if not
found)}\label{check-function-is-supposed-to-return-the-name-of-a-person-or-none-if-not-found}}

    \begin{tcolorbox}[breakable, size=fbox, boxrule=1pt, pad at break*=1mm,colback=cellbackground, colframe=cellborder]
\prompt{In}{incolor}{8}{\boxspacing}
\begin{Verbatim}[commandchars=\\\{\}]
\PY{c+c1}{\PYZsh{} Create a check function (to output str or None)}
\PY{k}{def} \PY{n+nf}{check}\PY{p}{(}\PY{n}{comp\PYZus{}data}\PY{p}{,} \PY{n}{database}\PY{p}{)} \PY{o}{\PYZhy{}}\PY{o}{\PYZgt{}} \PY{n+nb}{str} \PY{o+ow}{or} \PY{k+kc}{None}\PY{p}{:}

    \PY{c+c1}{\PYZsh{} Compare the lists (transform database to a list containing all the keys, omit \PYZsq{}name\PYZsq{})}
    \PY{k}{if} \PY{n}{comp\PYZus{}data} \PY{o}{==} \PY{n+nb}{list}\PY{p}{(}\PY{n}{database}\PY{o}{.}\PY{n}{values}\PY{p}{(}\PY{p}{)}\PY{p}{)}\PY{p}{[}\PY{l+m+mi}{1}\PY{p}{:}\PY{p}{]}\PY{p}{:}

        \PY{c+c1}{\PYZsh{} Return the name of the person}
        \PY{k}{return} \PY{n}{database}\PY{p}{[}\PY{l+s+s1}{\PYZsq{}}\PY{l+s+s1}{name}\PY{l+s+s1}{\PYZsq{}}\PY{p}{]}
    \PY{k}{else}\PY{p}{:}

        \PY{c+c1}{\PYZsh{} Otherwise return None}
        \PY{k}{return}
\end{Verbatim}
\end{tcolorbox}

    \hypertarget{open-selected-files-and-handle-possible-errors}{%
\subsubsection{Open selected files and handle possible
errors}\label{open-selected-files-and-handle-possible-errors}}

\textbf{Usage:}

\begin{Shaded}
\begin{Highlighting}[]
\ControlFlowTok{try}\NormalTok{:}
    \BuiltInTok{file} \OperatorTok{=} \BuiltInTok{open}\NormalTok{(path, }\StringTok{"\#"}\NormalTok{)}
\ControlFlowTok{except} \PreprocessorTok{OSError}\NormalTok{:}
\NormalTok{    file\_error(path)}
\end{Highlighting}
\end{Shaded}

\href{https://docs.python.org/3/tutorial/errors.html}{\textbf{Docs}}

    \begin{tcolorbox}[breakable, size=fbox, boxrule=1pt, pad at break*=1mm,colback=cellbackground, colframe=cellborder]
\prompt{In}{incolor}{9}{\boxspacing}
\begin{Verbatim}[commandchars=\\\{\}]
\PY{c+c1}{\PYZsh{} Open database file and handle possible errors}
\PY{k}{try}\PY{p}{:}
    \PY{n}{database\PYZus{}file} \PY{o}{=} \PY{n+nb}{open}\PY{p}{(}\PY{n}{database\PYZus{}path}\PY{p}{,} \PY{l+s+s2}{\PYZdq{}}\PY{l+s+s2}{r}\PY{l+s+s2}{\PYZdq{}}\PY{p}{)}
\PY{k}{except} \PY{n+ne}{OSError}\PY{p}{:}
    \PY{n}{file\PYZus{}error}\PY{p}{(}\PY{n}{database\PYZus{}path}\PY{p}{)}

\PY{c+c1}{\PYZsh{} Open sequence file and handle possible errors}
\PY{k}{try}\PY{p}{:}
    \PY{n}{sequence\PYZus{}file} \PY{o}{=} \PY{n+nb}{open}\PY{p}{(}\PY{n}{sequence\PYZus{}path}\PY{p}{,} \PY{l+s+s2}{\PYZdq{}}\PY{l+s+s2}{r}\PY{l+s+s2}{\PYZdq{}}\PY{p}{)}
\PY{k}{except} \PY{n+ne}{OSError}\PY{p}{:}
    \PY{n}{file\PYZus{}error}\PY{p}{(}\PY{n}{sequence\PYZus{}path}\PY{p}{)}
\end{Verbatim}
\end{tcolorbox}

    \hypertarget{file-structures}{%
\subsection{File structures}\label{file-structures}}

\hypertarget{database-file}{%
\subsubsection{Database file}\label{database-file}}

Custom database of type
\href{https://support.google.com/google-ads/answer/9004364?hl=en}{csv}
containg gathered people and their occurence of particular
\href{https://www.future-science.com/doi/10.2144/000112582}{STR(s)}.

\textbf{Structure:}

\begin{longtable}[]{@{}ccccc@{}}
\toprule
Name & STR1 & STR2 & \ldots{} & STRn \\
\midrule
\endhead
Name & n & n & \ldots{} & n \\
\bottomrule
\end{longtable}

*Where \texttt{n} represents the numerical occurence of a specific
\textbf{STR} in person's DNA sequence.

\hypertarget{sequence-file}{%
\subsubsection{Sequence file}\label{sequence-file}}

\textbf{DNA} is just a sequence of nucleotides arranged in a shape!
\href{ttps://www.nature.com/scitable/topicpage/dna-is-a-structure-that-encodes-biological-6493050/}{{[}1{]}}

\textbf{Structure:}

\begin{verbatim}
B1B2B3 ... Bn
\end{verbatim}

*Where \texttt{B} represents an indexed base of DNA's nucleotide.

\textbf{Possible nucleotides:}

\begin{longtable}[]{@{}cccc@{}}
\toprule
A & C & G & T \\
\midrule
\endhead
Adenine & Cytosine & Guanine & Thymine \\
\bottomrule
\end{longtable}

\href{https://www.genome.gov/genetics-glossary/Nucleotide}{Source}

    \begin{tcolorbox}[breakable, size=fbox, boxrule=1pt, pad at break*=1mm,colback=cellbackground, colframe=cellborder]
\prompt{In}{incolor}{10}{\boxspacing}
\begin{Verbatim}[commandchars=\\\{\}]
\PY{c+c1}{\PYZsh{} Continue with valid files}
\PY{k}{with} \PY{n}{sequence\PYZus{}file}\PY{p}{:}

    \PY{c+c1}{\PYZsh{} Read the dna sequence from the file (omit \PYZsq{}\PYZbs{}n\PYZsq{} and read only the first line)}
    \PY{n}{sequence} \PY{o}{=} \PY{n}{sequence\PYZus{}file}\PY{o}{.}\PY{n}{readlines}\PY{p}{(}\PY{p}{)}\PY{p}{[}\PY{l+m+mi}{0}\PY{p}{]}\PY{o}{.}\PY{n}{replace}\PY{p}{(}\PY{l+s+s2}{\PYZdq{}}\PY{l+s+se}{\PYZbs{}n}\PY{l+s+s2}{\PYZdq{}}\PY{p}{,} \PY{l+s+s2}{\PYZdq{}}\PY{l+s+s2}{\PYZdq{}}\PY{p}{)}

\PY{k}{def} \PY{n+nf}{repl}\PY{p}{(}\PY{n}{row}\PY{p}{)} \PY{o}{\PYZhy{}}\PY{o}{\PYZgt{}} \PY{n+nb}{dict}\PY{p}{:}
    \PY{l+s+sd}{\PYZdq{}\PYZdq{}\PYZdq{}}
\PY{l+s+sd}{    Dictionary comprehension if\PYZhy{}else structure}
\PY{l+s+sd}{    \PYZob{} (some\PYZus{}key if condition else default\PYZus{}key):(something\PYZus{}if\PYZus{}true if condition}
\PY{l+s+sd}{    else something\PYZus{}if\PYZus{}false) for key, value in dict\PYZus{}.items() \PYZcb{}}
\PY{l+s+sd}{    \PYZdq{}\PYZdq{}\PYZdq{}}

    \PY{c+c1}{\PYZsh{} Return each row from the csv file as a dict (and transform all str\PYZhy{}to\PYZhy{}int convertibles to ints)}
    \PY{n}{data}\PY{p}{:} \PY{n+nb}{dict} \PY{o}{=} \PY{p}{\PYZob{}}\PY{n}{key}\PY{p}{:} \PY{p}{(}\PY{n}{row}\PY{p}{[}\PY{n}{key}\PY{p}{]} \PY{k}{if} \PY{o+ow}{not} \PY{n}{row}\PY{p}{[}\PY{n}{key}\PY{p}{]}\PY{o}{.}\PY{n}{isdigit}\PY{p}{(}\PY{p}{)} \PY{k}{else} \PY{n+nb}{int}\PY{p}{(}\PY{n}{row}\PY{p}{[}\PY{n}{key}\PY{p}{]}\PY{p}{)}\PY{p}{)} \PY{k}{for} \PY{n}{key} \PY{o+ow}{in} \PY{n}{row}\PY{p}{\PYZcb{}}
    \PY{k}{return} \PY{n}{data}

\PY{c+c1}{\PYZsh{} Load data from the dna database and call the repl function}
\PY{k}{with} \PY{n}{database\PYZus{}file}\PY{p}{:}

    \PY{c+c1}{\PYZsh{} Populate list with respective rows of type dict}
    \PY{n}{dna\PYZus{}database}\PY{p}{:} \PY{n+nb}{list} \PY{o}{=} \PY{p}{[}\PY{n}{repl}\PY{p}{(}\PY{n}{row}\PY{p}{)} \PY{k}{for} \PY{n}{row} \PY{o+ow}{in} \PY{n}{csv}\PY{o}{.}\PY{n}{DictReader}\PY{p}{(}\PY{n}{database\PYZus{}file}\PY{p}{)}\PY{p}{]}

\PY{c+c1}{\PYZsh{} Store possible nucleotides (omit \PYZsq{}name\PYZsq{} key \PYZhy{}\PYZgt{} start from index 1)}
\PY{n}{nucleotides}\PY{p}{:} \PY{n+nb}{list} \PY{o}{=} \PY{p}{[}\PY{n}{key} \PY{k}{for} \PY{n}{key} \PY{o+ow}{in} \PY{n}{dna\PYZus{}database}\PY{p}{[}\PY{l+m+mi}{0}\PY{p}{]}\PY{p}{]}\PY{p}{[}\PY{l+m+mi}{1}\PY{p}{:}\PY{p}{]}

\PY{c+c1}{\PYZsh{} Search for chains of repeating nucleotides}
\PY{n}{computed\PYZus{}data}\PY{p}{:} \PY{n+nb}{list} \PY{o}{=} \PY{p}{[}\PY{n}{compute}\PY{p}{(}\PY{n}{sequence}\PY{p}{,} \PY{n}{nucleotide}\PY{p}{)} \PY{k}{for} \PY{n}{nucleotide} \PY{o+ow}{in} \PY{n}{nucleotides}\PY{p}{]}

\PY{c+c1}{\PYZsh{} Call the check and store its outputs in a list (only append if not of type None)}
\PY{n}{names}\PY{p}{:} \PY{n+nb}{list} \PY{o}{=} \PY{p}{[}\PY{n}{check}\PY{p}{(}\PY{n}{computed\PYZus{}data}\PY{p}{,} \PY{n}{dna}\PY{p}{)} \PY{k}{for} \PY{n}{dna} \PY{o+ow}{in} \PY{n}{dna\PYZus{}database} \PY{k}{if} \PY{n}{check}\PY{p}{(}\PY{n}{computed\PYZus{}data}\PY{p}{,} \PY{n}{dna}\PY{p}{)} \PY{o+ow}{is} \PY{o+ow}{not} \PY{k+kc}{None}\PY{p}{]}
\end{Verbatim}
\end{tcolorbox}

    \hypertarget{content-of-input-files}{%
\subsection{Content of input files}\label{content-of-input-files}}

In this section, \texttt{pandas} library will be used to display
individual data sets using a \texttt{DataFrame}.

\textbf{File structure(s)} described above.

\begin{itemize}
\tightlist
\item
  \href{https://pandas.pydata.org/docs/reference/api/pandas.DataFrame.html}{Pandas
  DataFrame}
\end{itemize}

Moreover, to display results of \textbf{DNA sequence} in
\texttt{Markdown}, \texttt{IPython} will be used.

\begin{itemize}
\tightlist
\item
  \href{https://ipython.readthedocs.io/en/stable/api/generated/IPython.display.html}{IPython.display}
\end{itemize}

    \begin{tcolorbox}[breakable, size=fbox, boxrule=1pt, pad at break*=1mm,colback=cellbackground, colframe=cellborder]
\prompt{In}{incolor}{11}{\boxspacing}
\begin{Verbatim}[commandchars=\\\{\}]
\PY{c+c1}{\PYZsh{} Import pandas lib.}
\PY{k+kn}{from} \PY{n+nn}{IPython}\PY{n+nn}{.}\PY{n+nn}{display} \PY{k+kn}{import} \PY{n}{display}\PY{p}{,} \PY{n}{HTML}
\PY{k+kn}{import} \PY{n+nn}{pandas} \PY{k}{as} \PY{n+nn}{pd}

\PY{c+c1}{\PYZsh{} Import IPython lib.}
\PY{k+kn}{from} \PY{n+nn}{IPython}\PY{n+nn}{.}\PY{n+nn}{display} \PY{k+kn}{import} \PY{n}{display}\PY{p}{,} \PY{n}{Markdown}

\PY{c+c1}{\PYZsh{} Create a dataframe}
\PY{n}{dt} \PY{o}{=} \PY{n}{pd}\PY{o}{.}\PY{n}{DataFrame}\PY{p}{(}\PY{n}{data}\PY{o}{=}\PY{n}{dna\PYZus{}database}\PY{p}{,} \PY{n}{index}\PY{o}{=}\PY{p}{[}\PY{n}{i} \PY{o}{+} \PY{l+m+mi}{1} \PY{k}{for} \PY{n}{i} \PY{o+ow}{in} \PY{n+nb}{range}\PY{p}{(}\PY{n+nb}{len}\PY{p}{(}\PY{n}{dna\PYZus{}database}\PY{p}{)}\PY{p}{)}\PY{p}{]}\PY{p}{)}
\PY{n}{display}\PY{p}{(}\PY{n}{Markdown}\PY{p}{(}\PY{l+s+s2}{\PYZdq{}}\PY{l+s+s2}{\PYZsh{}\PYZsh{}\PYZsh{}\PYZsh{} DNA database}\PY{l+s+s2}{\PYZdq{}}\PY{p}{)}\PY{p}{)}
\PY{n+nb}{print}\PY{p}{(}\PY{n}{dt}\PY{p}{,} \PY{n}{end}\PY{o}{=}\PY{l+s+s2}{\PYZdq{}}\PY{l+s+se}{\PYZbs{}n}\PY{l+s+s2}{\PYZdq{}} \PY{o}{*} \PY{l+m+mi}{3}\PY{p}{)}

\PY{c+c1}{\PYZsh{} Display sequence in markdown}
\PY{n}{display}\PY{p}{(}\PY{n}{Markdown}\PY{p}{(}\PY{l+s+sa}{f}\PY{l+s+s2}{\PYZdq{}}\PY{l+s+s2}{\PYZsh{}\PYZsh{}\PYZsh{}\PYZsh{} DNA sequence}\PY{l+s+se}{\PYZbs{}n}\PY{l+s+s2}{`}\PY{l+s+se}{\PYZbs{}n}\PY{l+s+si}{\PYZob{}}\PY{n}{sequence}\PY{l+s+si}{\PYZcb{}}\PY{l+s+se}{\PYZbs{}n}\PY{l+s+s2}{`}\PY{l+s+s2}{\PYZdq{}}\PY{p}{)}\PY{p}{)}
\end{Verbatim}
\end{tcolorbox}

    \hypertarget{dna-database}{%
\paragraph{DNA database}\label{dna-database}}

    
    \begin{Verbatim}[commandchars=\\\{\}]
     name  AGATC  TTTTTTCT  AATG  TCTAG  GATA  TATC  GAAA  TCTG
1  Daniel     14        44    28     27    19     7    25    20
2    Fero     29        29    40     31    45    20    40    35
3  Michal      6        18     5     42    39    28    44    22
4    Adam     37        47    13     25    17     6    13    35
5  Rachel     29        27    32     41     6    27     8    34
6    Mike     31        11    28     26    35    19    33     6


    \end{Verbatim}

    \hypertarget{dna-sequence}{%
\paragraph{DNA sequence}\label{dna-sequence}}

\begin{Shaded}
\begin{Highlighting}[]
\ExtensionTok{GGGTGAATCTCGGACAAAAGGGACCCAGTAATGGGAAAACACCCTGTACTTTCATTTATCTTAAGGAAAAGGCACTGCGACTTGTGACTGTTTTGTGCTTACCGGTCTGCTGACCACCTCCGCAAGATTCACCGGGCCCTCGCCCCTGGGCCCGCGGGGCTGTTTCCCCATTAATTCTGCACGGGCGAAGGCGGCCCCTCGGCCCGATAACGGACGTTGAAAGGCACCCAGTGTACCAACTCTATCGTTGTTAATCTCTAGTCTAGTCTAGTCTAGTCTAGTCTAGTCTAGTCTAGTCTAGTCTAGTCTAGTCTAGTCTAGTCTAGTCTAGTCTAGTCTAGTCTAGTCTAGTCTAGTCTAGTCTAGTCTAGTCTAGTCTAGTCTAGTATTACAGCGGCATCACTATACTATAATAGTCCGACATGTTAGAATTCTGACGCGTCGAACCGTAGGCGAGTCCGAGTTTCCTACTCCTCACGTGGTCAGAAGTCCCGCTCGAGGATACCGATAAGCCTGAATTCTTCATTTCTACTAACTCGACAGACTGACTCACGCTAGTTTGGTTACTGTCCCGTACCCGCGTTTTCCACCTTAGTCTTGCTGGCGTACACTTTCAAGCGATAGCATGCTTACAATAGGCTATTGCGACGTGTCCGATCCTACATGTACTCATATATCGCAACGGCCCAGGTTACTTAGGGACTAAACGGCCCTTTAAAGCGAGGGTAGTAGAATTCAGGCCACTGAAATTGGGATTATCTAATAAATCACCCGCCGCCGAGATAAGGCAGCTTACGAGTGAAGTTTGCATCACGTGCCTTTGTATTATTAATCTCGACTACCAACTCCTCAATTATAAGCAAGTTCGTTCATAAATAGTCAGATGCTTGGGGCGCATGCGGTATTGGAGTTGCCATGGTCTACACGCGGCCCTATGCAAACTTTTCTTAAGCGAGGAAGTCTTCCGTATTGCGTTGCATTTTCTCAATTGTTTTAGTCTTGTCGTAACCAATGCGGTAAGAATGCGTTTACAGGGCCGCGACCCCAACAATCCTTCCTTCGAGGAAGGTACTGAGTAGGGCTGGTCTATGCCGGTACTCCTTAGGTTGTTCACTCATGCCGATACCACTCTTGGTTTCTCACCCTTATTCTGGCACTGGTAAAAGATAAAGAGAACCTCCAGTCTAAGCGATTTACTTGGACACGCCCTTTTTGAAGTTCGGGCACACTCCGATTACCTAGAGCCCTAATGGGAGCCCGTAATTTCCAATAAAAGGAAAGTGGCTATCAAGCCATCCGCTCGAAGCACGTCATCCGGTACTTAGGTGTCCCTTGTGTCGCTTAATTAGATGTTGTACGAACGATATATACGGGCTAGGATAGATAGATAGATAGATAGATAGATAGATAGATAGATAGATAGATAGATAGATAGATAGATAGATAGATAGATAGATAGATAGATAGATAGATAGATAGATAGATAGATAGATAGATAGATAGATAGATAGATAGATACGTCCGGCATCCAGCTCGTCAATCTTTAAGTGGCGTCATCAGTAAGTGCACGTGGTCTTCCAAACCCTGTGGCGACACTATTAGACGTGGTCGGATCGGTCTTTATCTTTGGTTCTGTAGTAGGAGAACTCATGGGATGGTTATGGAGTTTGACTACAGCGTCTGGTCTCGGTAGTCCCCACACCCTCTAGCCCAACCACTAGAAAGTGACTAGTAATATGCGTTCTGCGAATCTCAGGGGTCAACCGTGACTATTGATCTTACCAGCGCATCGCTCCGCAAGCTCAGTAAATTTGATGTTTTTCCACAACCGAAAGAGCGATTAAGTACTGAACAAGAGTGTTAAGTCCAGTACTAACAATTAGATACATACTCGACTCGTAGAATCTTTCGTATTCCTGGTCGTACCCAGCAACGCCAGCAGAGTTTAGTCGACTGGTAGTTTGGAGTTTTCGAGGACGGAGACTGTGCATAAACTGTAGACATTCTGAAGCGCATGTCGGAGTGTATTCACGTGCACGCTCACCATGAGTTCAGGTCTATGGTGTCGGCGTTCACAAGTCCATAATCTGCGTCTCGTGACGTGACTTATCCCTTGTCACTCTAGTTGGACATCACAGTAGCGTTCGCCTCCTTCGTAATTCCTATCCCCAAAGAGAGTATTTCAATTAATACTCTAAACTCGAGCGCCGAGCGACACCATTCCATTCTAAATCTGGGGGCTCACGCGCTTTCAGTCGTCAATTTTATCGCGAAATAAACATGATCGCTATAAGATATCCCTTCCTCATCTGTTGCCCCAACCTGAGGCGTTCTAGACGTACACGATTAACGCGTTACGCGACAGCAGAATCGAGCTACGTGCAGAGCGTATCTGCGCCGGGGATCGCCGTAGAAGACGGCGCTAGCCAACCAGGGTTCGACGAACTGCCCTAGGAAACAAGCAAGCATGCTCGTGAGAGCAGGACCTTCTGCAGAAGATATTGCGATACATGAGTTTCTACGGTATTGGGTTATGGAATCTAAGAGGGCCAATGGAAATAGTAAGTTGGGGCGGATTATTTGAATACCGTTCGGTGGACTGTTTTCCGGAAGAGCGATGCCACCTCCTTGGTGTCTGCATCGAATAGATGCCCTGTCTTTCGTAATCGGTGAACCCATTAGTGACTCATTCGTGGCCGATACATTTACTTATACGTCTTGAGGCGCAGCAAGTCTAGTATGTCGTATAGAAAGCAGGTTTGCTTAGTTCGACTTAAGAGTGACGCTAGGTCACGAATTTCTCGTCCGGGGCATGTCAGAGATTGATTCCTTAGATACTGGCCAAACCGATACTACGCGTTGATAAGTGAACGACATGAAAGTTGACAACACCCTCTAGGGTTCGAACCAAAACAGAAAGTAGGAGCAACGTTCTCGGAGCACCACATCACTTGAACCGCCACTTCGACTTTCACCGAGTGACATAAAGACTAGTGGACCCCATCTGTTACCAATAAGGGGCGTGTAGATCAGCACTGGAATACAACACACGATGCCTCAGTCATCTCACTACGTGTCTCCTCGCTCCGCGTTCGCACTATTGCTCGCGCCCACCGCACACTACCGTGTTGAATGACGAAAGTTGCGGGGCTTACGCTCGGTGCAAAAGTTTCACATCTCTTGATAGCATTGGGCAACGCCGACGTATAGCTATGCAATTCACACGGCCGCAAAGCTATTTGACTGAACGTCTAGCATAGTGGTCAAACTGGCTCGAGCTAGATCTTGACGCACGTTCGGTATCTTGGAATTACCACCGGTACATGAGAGCAAAAGGGGAATCCTTGGTATGGAATTTACAAACCTGATCATCTTATTAGTGGTGTGATGAGATATCTTCATCTTTATAAACGTGGCGCGGCCACTTAGCGCTCAACGGCAATGAATGAATGAATGAATGAATGAATGAATGAATGAATGAATGAATGAATGAATGAATGAATGAATGAATGAATGAATGAATGAATGAATGAATGAATGAATGAATGAATGTTTTTTCTTTTTTTCTTTTTTTCTTTTTTTCTTTTTTTCTTTTTTTCTTTTTTTCTTTTTTTCTTTTTTTCTTTTTTTCTTTTTTTCTCCGTCGGCTCTGCGACGCGCGGAGTCAGTCTTCTGCCCCGGGGTCAGCCAGTCAAGACCTGCTGGGAAACGAAGAGAAAGAACAGTGCTAATGCAGGCTCCTCTGACGTCTATGTACTATAGCTTTTCCGCACGCGGAATCTGTCATTGCTACAAGCCGTGCCCAGGGCGAATATTCTGGCCGCACCAAAAGGAAAGAAAGAAAGAAAGAAAGAAAGAAAGAAAGAAAGAAAGAAAGAAAGAAAGAAAGAAAGAAAGAAAGAAAGAAAGAAAGAAAGAAAGAAAGAAAGAAAGAAAGAAAGAAAGAAAGAAAGAAAGAAAGAAACGGGGCTTTGCCGTCAATAGCACGAGCTGAGACGGGGAAGCCCGTATTTCATACAGGTCACGTTCAAGAGACAGCGCATTAAAGATCATGCGCGGAAGGGTTAGCAAGGTCGCAAAAGGAAACATTGTCTTTTTCAAAGGGGCCTCTGCCCCAGATCAGATCAGATCAGATCAGATCAGATCAGATCAGATCAGATCAGATCAGATCAGATCAGATCAGATCAGATCAGATCAGATCAGATCAGATCAGATCAGATCAGATCAGATCAGATCAGATCAGATCAGATCAGATCAGATCAGATCAGATCCAGATAAACACATGTCTTGCGTTTTCTCGCCTGTTACTGGCTCGGCTCGCCTATGTCTTTTATATAAAACTATAAGGGTGTCGTCTGCTCCGTTTGGTCGGCTGAACGAACCTATGTTCTGTCTTTACCACACTTTCTCCGTTGCAAGGCTCAACGCCCCCAATTTTAAAATGTTCGAGATTCTGTCTGTCTGTCTGTCTGTCTGAAGTTAATAACCGAAACGATCTAATTTACAGTAGACATGGGCGAATGGCGTAAGCAGACAGTGACCGATAACGCTTCCTCCGCGCAGTTGGCCCCGGCTGACTCAAGAGCGGTACGGTCGGCCTTCTTGGCCATAGCCTACACACATCTCAAATGTCGCAAAGCGAACTACCTGCACCGGAGTCCGGGAATCTACTGCACGTTGGGCCTCTTGTACTGATTGCTGCCCGCTACGAACTAGAGCATAATAAAACTTCCGCCGAACCATTGGACTGAGATCAGAGGTAGCTTGATATGGGATCAGAGTTAGTCGATCCAAGTGAAGCTTGTAAGCCAATAGATTGAACATAGGATCATATTACCCCCACCGGTCCCACATTGCTGGGACGGACCCTGTCAGAGAGAGATCTAAACATCATTTGGACTTTGCCTATCTGCGGGACACCACAGGAGCTCCCAGCGCACGAAAAGAGCGGATTCACTAATCCTCTGGCAGTCCTCTTGAGGCATGGTGGGGATGGGTAATTAAGTCGAGGCCGTAGCACGGGATCGGCAACACAAGAATGGCACAAAACATATTCCCTTAGCTCCAAGATGTCGTAGTAGTAGTAGGATGAGAGGCGTGTCATGGGTTCCGTAGTATAGTGCATGCCAACTGTCTTGAATGCTAATATAGATCTTGGAGGAAGCTGGATCGTTGGTCCCTCCGACAGACGATTAAATAGTGGCGTAGATCGCACAAAGTTGCATTAGCAAGCGGGGCGTATCTAGAGTTTACTGTGCTGTAGGATCGGCGCGCTAGCAACAAAAGATTATCCACACAACAGACGCCCTCTCATGTCATAGACCGTGGGGGGGTCCGCACTTTTGGATTGCTGCTATACTATAAGACTCGTCTATCAGAAAACTTCTCGATGGTCCGGAGCCTCAGGACACCCCCCTCACCCATCGCCGGGCAAATGGCTCACCTCGATAATTTTAATCAACGGTGAACATCCAATTCCACACGGAGAGAACCGTACGTCTCAAGATCACTGCTTGTTTAAAATTGCAGAGTAGACCGATTCCGTACTTTAGAATCCAGCGATCCCTCAGCATGTCCATTTGCTCGACTTAATCATGCTGGTATCTATCTATCTATCTATCTATCTATCTATCTATCTATCTATCTATCTATCTATCTATCTATCTATCTATCTATCGCCTCGCTATCGACGTCGCATACATGCTTAGTTCTTGCCCAAGTTGACTGACGATCTTAGTCATAGTCTCACACCTGATA}
\end{Highlighting}
\end{Shaded}

    
    \begin{tcolorbox}[breakable, size=fbox, boxrule=1pt, pad at break*=1mm,colback=cellbackground, colframe=cellborder]
\prompt{In}{incolor}{12}{\boxspacing}
\begin{Verbatim}[commandchars=\\\{\}]
\PY{c+c1}{\PYZsh{} If the list is not of size 1 \PYZhy{}\PYZgt{} no match}
\PY{k}{if} \PY{n+nb}{len}\PY{p}{(}\PY{n}{names}\PY{p}{)} \PY{o}{!=} \PY{l+m+mi}{1}\PY{p}{:}
    \PY{n+nb}{print}\PY{p}{(}\PY{l+s+s2}{\PYZdq{}}\PY{l+s+s2}{No match}\PY{l+s+s2}{\PYZdq{}}\PY{p}{)}
\PY{k}{else}\PY{p}{:}

    \PY{c+c1}{\PYZsh{} Otherwise only 1 valid person was found \PYZhy{}\PYZgt{} print them out (in Markdown)}
    \PY{n}{display}\PY{p}{(}\PY{n}{Markdown}\PY{p}{(}\PY{l+s+sa}{f}\PY{l+s+s2}{\PYZdq{}}\PY{l+s+s2}{\PYZsh{}\PYZsh{}\PYZsh{}\PYZsh{} Person}\PY{l+s+s2}{\PYZsq{}}\PY{l+s+s2}{s name: }\PY{l+s+si}{\PYZob{}}\PY{n}{names}\PY{p}{[}\PY{l+m+mi}{0}\PY{p}{]}\PY{l+s+si}{\PYZcb{}}\PY{l+s+s2}{\PYZdq{}}\PY{p}{)}\PY{p}{)}
\end{Verbatim}
\end{tcolorbox}

    \hypertarget{persons-name-mike}{%
\paragraph{Person's name: Mike}\label{persons-name-mike}}

    

    % Add a bibliography block to the postdoc
    
    
    
\end{document}
